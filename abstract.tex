\chapter*{Introducción}
\addcontentsline{toc}{chapter}{Abstract}


La inteligencia de enjambre (\textit{Swarm Intelligence}, SI) es un paradigma emergente en el campo de la inteligencia artificial inspirado en el comportamiento colectivo observado en sistemas biológicos, como colonias de hormigas, enjambres de abejas y bandadas de aves. Estos sistemas exhiben propiedades emergentes, donde interacciones simples entre agentes individuales generan soluciones robustas y eficientes para problemas complejos. 

Los algoritmos basados en SI, como la Optimización por Enjambre de Partículas (\textit{Particle Swarm Optimization}, PSO) \cite{eberhart1995new}, la Optimización basada en Colonias de Hormigas (\textit{Ant Colony Optimization}, ACO) \cite{dorigo1996ant}, y el Algoritmo de Búsqueda Cuco (\textit{Cuckoo Search Algorithm}, CSA) \cite{yang2009cuckoo}, han demostrado ser herramientas efectivas en la resolución de problemas de optimización global, diseño de redes, y robótica cooperativa.

A diferencia de los métodos tradicionales de inteligencia artificial, la SI no requiere de un controlador centralizado o una gran cantidad de datos para entrenar modelos, lo que la hace adecuada para sistemas distribuidos y entornos dinámicos. Este artículo explora las características fundamentales de la inteligencia de enjambre, describe los algoritmos más representativos, y analiza sus aplicaciones en diversos campos, destacando su potencial para abordar problemas computacionalmente desafiantes.

